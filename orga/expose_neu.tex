# Entwicklung eines Netzwerkprüfgeräts

Im Rahmen dieser Bachelorarbeit soll ein Gerät entwickelt werdern, welches dazu verwendet werden kann, um festzustellen ob ein Ethernetport Zugang zu einem Netzwerk erlaubt. Hierzu soll das Gerät über ein LAN-Kable an einen Ethernetport angeschlossen werden, ein DHCP Discover durchführen und versuchen eine IP Addresse zu erhalten.  
Wenn es Zugriff auf das Netz hat, soll das Gerät Informationen zum Netzwerk anzeigen, wie Netzbereich, aber auch statistische Informationen über sichtbare Protokolle.  
Ein Touchscreen soll sowohl als Eingabegerät zur Bedienung, als auch als Ausgabegerät für Informationen dienen.  
Das Gerät soll auf Benutzerfreundlichkeit und geringen Konfigurationsaufwand von einem Benutzer optimiert werden.  
In einer Literaturreschersche soll nach bereiets vorhanden Lösungen gesucht werden, um mögliche Lösungsansätze miteinander zu vergleichen.  
Software, entwickelt im Rahmen des Projektes, soll unter einer openSource Lizenz verwaltet werden.
Des Weiteren wird in dieser Bachelorarbeit geprüft, ob und wie DHCP implementiert werden kann und muss. 

Ziel der Bachelorarbeit ist es zu überprüfen, ob es möglich ist das beschreibene Gerät zu entwickeln. Es soll weitere folgende Anforderungen erfüllen:  
* Das Gerät soll sich in einem Finanziellen Rahmen unter 300 Euro befinden.  
* Für die Hardwaretechnische Umsetzung soll ein Raspberry Pi und Touchscreen, mit einer entsprechenden Energiequelle, verwendet werden, sodass ein tragbares Testgerät entsteht.  
* Das Gerät soll mit einem, für Bachelorarbeiten typischen, Zeitaufwand entwickelt werden.  
* Es soll ein Anwendung mit einem UserInterface für ein Touchscreen entwickelt werden.  
* Das Gerät soll versuchen einen DHCP Lease zu bekommen dies in dem eben genannten UI darzustellen.  * Die Anwendung sollte im Kioskmode starten.  

Wenn es der Zeitliche Rahmen erlaubt, können folgende weitere Sicherheitsfeatures implemetniert werden:  
* Ping Sweep, durch einen Ping Sweep könnte ein schneller Überblick, über das evtl. freiliegende Netz geschaffen werden.  
* MAC Masking, hierbei wird eigene MAC Addresse auf, die eines bereits im Netz befindlichen Gerätes, gesetzt. Mit dieser Methodik könnte geprüft werden ob der Port, Angriffe auf die Portsecurity, erkennt.
